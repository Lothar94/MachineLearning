%%%%%%%%%%%%%%%%%%%%%%%%%%%%%%%%%%%%%%%%%%%%%%%%%%%%%%%%%%%%%%%%%%%%%%
% LaTeX Template: Beamer arrows
%
% Source: http://www.texample.net/
% Feel free to distribute this template, but please keep the
% referal to TeXample.net.
% Date: Nov 2006
% 
%%%%%%%%%%%%%%%%%%%%%%%%%%%%%%%%%%%%%%%%%%%%%%%%%%%%%%%%%%%%%%%%%%%%%%
% How to use writeLaTeX: 
%
% You edit the source code here on the left, and the preview on the
% right shows you the result within a few seconds.
%
% Bookmark this page and share the URL with your co-authors. They can
% edit at the same time!
%
% You can upload figures, bibliographies, custom classes and
% styles using the files menu.
%
% If you're new to LaTeX, the wikibook is a great place to start:
% http://en.wikibooks.org/wiki/LaTeX
%
%%%%%%%%%%%%%%%%%%%%%%%%%%%%%%%%%%%%%%%%%%%%%%%%%%%%%%%%%%%%%%%%%%%%%%

\documentclass{beamer} %
\usetheme{CambridgeUS}
\usepackage[utf8]{inputenc}
\usefonttheme{professionalfonts}
\usepackage{times}
\usepackage{tikz}
\usepackage{amsmath}
\usepackage{verbatim}
\usetikzlibrary{arrows,shapes}

\author{Daniel López García\\ Lothar Soto Palma\\ Elena María Toro Pérez\\\textit{Universidad de Granada}}
\title{Historia de las matemáticas\\ Machine Learning}

\begin{document}

% For every picture that defines or uses external nodes, you'll have to
% apply the 'remember picture' style. To avoid some typing, we'll apply
% the style to all pictures.
\tikzstyle{every picture}+=[remember picture]

% By default all math in TikZ nodes are set in inline mode. Change this to
% displaystyle so that we don't get small fractions.
\everymath{\displaystyle}

\begin{frame}
\titlepage
\end{frame}

\begin{frame}
\frametitle{Índice}
\tableofcontents
\end{frame}

\section{Introducción}
	\begin{frame}
	\frametitle{Introducción}
	\end{frame}

\section{Tipos de aprendizaje}
	\begin{frame}
	\frametitle{Tipos de aprendizaje}
	\end{frame}

\section{Problema de clasificación}
	\begin{frame}
	\frametitle{Problema de Clasificación}
	\end{frame}
	
	\subsection{Redes neuronales}
		\begin{frame}
		\frametitle{Redes neuronales}
		\end{frame}
	\subsection{Vecino más cercano}
		\begin{frame}
		\frametitle{Vecino más cercano}
		\end{frame}
	\subsection{Árboles de decisión}
		\begin{frame}
		\frametitle{Árboles de decisión}
		\end{frame}
	\subsection{SVM}
		\begin{frame}
		\frametitle{SVM}
		\end{frame}

\section{Otras técnicas para el aprendizaje}
	\begin{frame}
	\frametitle{Otras técnicas para el aprendizaje}
	\end{frame}

	\subsection{Clustering}
		\begin{frame}
		\frametitle{Clustering}
		\end{frame}

	\subsection{Algoritmos genéticos}
		\begin{frame}
		\frametitle{Algoritmos genéticos}
		\end{frame}

	\subsection{Colonias de hormigas}
		\begin{frame}
		\frametitle{Colonias de hormigas}
		\end{frame}

\section{Máquinas y juegos}
	\begin{frame}
	\frametitle{Máquinas y juegos}
	\end{frame}


\end{document}
              
            